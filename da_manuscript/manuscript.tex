% Options for packages loaded elsewhere
\PassOptionsToPackage{unicode}{hyperref}
\PassOptionsToPackage{hyphens}{url}
\PassOptionsToPackage{dvipsnames,svgnames,x11names}{xcolor}
%
\documentclass[
  lineno]{asm}

\usepackage{amsmath,amssymb}
\usepackage{iftex}
\ifPDFTeX
  \usepackage[T1]{fontenc}
  \usepackage[utf8]{inputenc}
  \usepackage{textcomp} % provide euro and other symbols
\else % if luatex or xetex
  \usepackage{unicode-math}
  \defaultfontfeatures{Scale=MatchLowercase}
  \defaultfontfeatures[\rmfamily]{Ligatures=TeX,Scale=1}
\fi
\usepackage{lmodern}
\ifPDFTeX\else  
    % xetex/luatex font selection
\fi
% Use upquote if available, for straight quotes in verbatim environments
\IfFileExists{upquote.sty}{\usepackage{upquote}}{}
\IfFileExists{microtype.sty}{% use microtype if available
  \usepackage[]{microtype}
  \UseMicrotypeSet[protrusion]{basicmath} % disable protrusion for tt fonts
}{}
\usepackage{xcolor}
\setlength{\emergencystretch}{3em} % prevent overfull lines
\setcounter{secnumdepth}{-\maxdimen} % remove section numbering
% Make \paragraph and \subparagraph free-standing
\makeatletter
\ifx\paragraph\undefined\else
  \let\oldparagraph\paragraph
  \renewcommand{\paragraph}{
    \@ifstar
      \xxxParagraphStar
      \xxxParagraphNoStar
  }
  \newcommand{\xxxParagraphStar}[1]{\oldparagraph*{#1}\mbox{}}
  \newcommand{\xxxParagraphNoStar}[1]{\oldparagraph{#1}\mbox{}}
\fi
\ifx\subparagraph\undefined\else
  \let\oldsubparagraph\subparagraph
  \renewcommand{\subparagraph}{
    \@ifstar
      \xxxSubParagraphStar
      \xxxSubParagraphNoStar
  }
  \newcommand{\xxxSubParagraphStar}[1]{\oldsubparagraph*{#1}\mbox{}}
  \newcommand{\xxxSubParagraphNoStar}[1]{\oldsubparagraph{#1}\mbox{}}
\fi
\makeatother


\providecommand{\tightlist}{%
  \setlength{\itemsep}{0pt}\setlength{\parskip}{0pt}}\usepackage{longtable,booktabs,array}
\usepackage{calc} % for calculating minipage widths
% Correct order of tables after \paragraph or \subparagraph
\usepackage{etoolbox}
\makeatletter
\patchcmd\longtable{\par}{\if@noskipsec\mbox{}\fi\par}{}{}
\makeatother
% Allow footnotes in longtable head/foot
\IfFileExists{footnotehyper.sty}{\usepackage{footnotehyper}}{\usepackage{footnote}}
\makesavenoteenv{longtable}
\renewcommand{\figurename}{FIG}
\usepackage{orcidlink}
\usepackage{blindtext}
\usepackage{chemformula}
\usepackage{siunitx}

\definecolor{mypink}{RGB}{219, 48, 122}
\makeatletter
\@ifpackageloaded{caption}{}{\usepackage{caption}}
\AtBeginDocument{%
\ifdefined\contentsname
  \renewcommand*\contentsname{Table of contents}
\else
  \newcommand\contentsname{Table of contents}
\fi
\ifdefined\listfigurename
  \renewcommand*\listfigurename{List of Figures}
\else
  \newcommand\listfigurename{List of Figures}
\fi
\ifdefined\listtablename
  \renewcommand*\listtablename{List of Tables}
\else
  \newcommand\listtablename{List of Tables}
\fi
\ifdefined\figurename
  \renewcommand*\figurename{Figure}
\else
  \newcommand\figurename{Figure}
\fi
\ifdefined\tablename
  \renewcommand*\tablename{Table}
\else
  \newcommand\tablename{Table}
\fi
}
\@ifpackageloaded{float}{}{\usepackage{float}}
\floatstyle{ruled}
\@ifundefined{c@chapter}{\newfloat{codelisting}{h}{lop}}{\newfloat{codelisting}{h}{lop}[chapter]}
\floatname{codelisting}{Listing}
\newcommand*\listoflistings{\listof{codelisting}{List of Listings}}
\makeatother
\makeatletter
\makeatother
\makeatletter
\@ifpackageloaded{caption}{}{\usepackage{caption}}
\@ifpackageloaded{subcaption}{}{\usepackage{subcaption}}
\makeatother

\usepackage{bookmark}

\IfFileExists{xurl.sty}{\usepackage{xurl}}{} % add URL line breaks if available
\urlstyle{same} % disable monospaced font for URLs
\hypersetup{
  pdftitle={This Is the Sample Article Template for mSystems , an American Society for Microbiology (ASM) Journal},
  pdfauthor={First Author; Second Author; Third Author; Third Final Author},
  pdfkeywords={keyword 1, keyword 2, keyword 3.},
  colorlinks=true,
  linkcolor={blue},
  filecolor={Maroon},
  citecolor={Blue},
  urlcolor={blue},
  pdfcreator={LaTeX via pandoc}}



\title{This Is the Sample Article Template for \emph{mSystems}
\textsuperscript{\textregistered}, an American Society for Microbiology
(ASM) Journal}

  \author[1]
  {First Author \orcidlink{0000-0000-0000-0000}}
  \author[1]
  {Second Author }
  \author[2]
  {Third Author }
  \author[1,2,\textdagger]
  {Third Final Author }

\affil[1]{University Name, Faculty Group, Department, City, Country}
\affil[2]{Company Name, City, Country}

  \corraddress{fourth@author.edu}

% paper meta
\papertype{Research Article}
\runningtitle{Running Title}
\runningauthor{FirstAuthor \textit{et al.}}
\equalcontrib{First Author and Second Author contributed equally to this
work. Author order was determined XXXXXX.}

\begin{document}

\maketitle

\begin{abstract}
Research Articles have structured abstracts consisting of two sections
with their own headings; ``Abstract'' and ``Importance''. Because the
structured abstract will be published separately by abstracting
services, it must be complete and understandable without reference to
the text. The Abstract section should be no more than 250 words and
should concisely summarize the basic content of the paper without
presenting extensive experimental details.
\begin{importance}
The Importance section should be no more than 150 words and should
provide a nontechnical explanation of the significance of the study to
the field. Avoid abbreviations and references, and indicate the specific
organism under study. When it is essential to include a reference, use
the format shown under ``References'' below.
\end{importance}
\end{abstract}

\keywords{keyword 1, keyword 2, keyword 3.}
\infobox

\renewcommand{\figurename}{FIG}
\renewcommand{\tablename}{TABLE}

\dropcap{P}lease read the
\href{https://journals.asm.org/journal/msystems/submission-review-process}{Instructions
to Authors} carefully, or browse the
\href{https://journals.asm.org/journal/msystems/faq}{FAQs} for further
details.

\section{Introduction}\label{sec-intro}

The introduction should supply sufficient background information to
allow the reader to understand and evaluate the results of the present
study without referring to previous publications on the topic. The
introduction should also provide the hypothesis that was addressed or
the rationale for the present study. Choose references carefully to
provide the most salient background rather than an exhaustive review of
the topic.

\begin{figure}

\centering{

\pandocbounded{\includegraphics[keepaspectratio]{style-guide/example-image-16x9.png}}

}

\caption{\label{fig-example}This is an example figure with caption. Use
the fullwidth environment to make it span the entire width of the page.
Lorem ipsum dolor sit amet, consectetur adipiscing elit.}

\end{figure}%

\subsection{Sectioning commands.}\label{sectioning-commands.}

Use \texttt{\#} to get a first-level heading. You can use \texttt{\#\#}
to get a sub-heading. Further sectioning levels, such as
\texttt{\#\#\#}, etc., are ignored.

Sections \textbf{must} be ordered as follows:

\begin{itemize}
\tightlist
\item
  Abstract
\item
  Importance

  \begin{itemize}
  \tightlist
  \item
    publishing data helps get more use out of research
  \item
    helps eliminate file drawer effect as it shows more negative data
  \item
    want to incentivize authors to publish/make available their original
    data
  \end{itemize}
\item
  Keywords

  \begin{itemize}
  \tightlist
  \item
    data accessibility
  \item
    data reproducibility
  \end{itemize}
\item
  Introduction

  \begin{itemize}
  \tightlist
  \item
    NIH funded research must make data available as of January 2023
    (Policy for Data Management and Sharing (NOT-OD-21-013))
  \item
    investigate metrics of making data publicly available in 12 ASM
    journals
  \item
    DNA sequencing efforts are commonly uploaded to databases

    \begin{itemize}
    \tightlist
    \item
      want to evaluate how well this community is using reporducible
      data analysis
    \end{itemize}
  \end{itemize}
\item
  Results
\item
  Discussion
\item
  Materials and Methods

  \begin{itemize}
  \tightlist
  \item
    oringinal dataset from adena (which i think is from crossref)
  \item
    hand identifying 500 papers for da and nsd status
  \item
    training model using mikropml methodology
  \item
    picking a model (glmnet, rf, xgbtree, picked rf)
  \item
    training of the models
  \item
    hypertuning parameters (rf = mtry)
  \item
    Snakemake/python
  \item
    crossref gathering of DOIs
  \item
    webscrape using httr2/rcrossref/wget
  \item
    cleaning of html of each indiviudal file
  \item
    tokenizing, stemming/lemmitization
  \item
    formatting/applying zscore (and replicating that for the rest of the
    datasets)
  \item
    using the model to predict da/nsd
  \end{itemize}
\item
  Supplemental Material file list (where applicable)
\item
  Acknowledgments
\item
  References
\end{itemize}

\subsection{Citations and References.}\label{citations-and-references.}

This template uses BibTeX and natbib, so \verb|\citep| and \verb|\citet|
or \texttt{@{[}citekey{]}} such as \citep{caserta:etal:2012},
\citet{johnson:robinson:2016}, or \citep{winnick:etal:2005} can be used
as usual to produce the correct citation style, and the reference list
is generated automatically. In the reference list, references are
numbered in the order in which they are cited in the article
(citation-sequence reference system). In the text, references are cited
parenthetically by number in sequential order. Data that are not
published or not peer reviewed are simply cited parenthetically in the
text. The
\href{https://journals.asm.org/journal/msystems/article-types}{mSystems
Instructions to Authors} contain additional guidelines for the reference
list and examples of how various types of references should be presented
in the manuscript.

\begin{enumerate}
\def\labelenumi{\roman{enumi}.}
\item
  \textbf{References listed in the References section.} The mSystems
  Instructions to Authors contain examples of the types of items that
  should be cited in the reference list. Those examples have been
  reproduced in the template bib file, as have other sample references.
\item
  \textbf{References cited in the text.} As indicated in the mSystems
  Instructions to Authors, certain reference types should be cited
  parenthetically rather than in the reference list. This paragraph
  shows examples of reference citations as they might appear within the
  manuscript text instead of in the reference list. A citation of
  unpublished data should appear as shown here (R.\textasciitilde B.
  \textasciitilde Layton and C.\textsubscript{C.}Weathers, unpublished
  data). A citation of a manuscript submitted for publication should
  appear as shown here (J.\textsubscript{L.}McInerney,
  A.\textsubscript{F.}Holden, and P.\textsubscript{N.}Brighton,
  submitted for publication). Citations of nonpublished abstracts and
  posters, etc., should appear as shown here (M.\textsubscript{G.}Gordon
  and F.\textsubscript{L.}Rattner, presented at the Fourth Symposium on
  Food Microbiology, Overton, IL, 13 to 15 June 1989). For
  non-U.S.\textasciitilde patent applications, give the date of
  publication of the application, as shown here
  (V.\textsubscript{R.}Smoll, 20 June 1999, Australian Patent Office). A
  website should be cited as shown here for the World Health
  Organization
  (\url{https://www.who.int/news-room/detail/17-01-2020-lack-of-new-antibiotics-threatens-global-efforts-to-contain-drug-resistant-infections}).
  URLs for companies that produce any of the products mentioned in your
  study or for products being sold may not be included in the article.
  However, company URLs that permit access to scientific data related to
  the study or to shareware used in the study are permitted.
\item
  \textbf{Citations in abstracts.} Since the abstract must be able to
  stand apart from the article, references cited in it should be clear
  without recourse to the References section. Use an abbreviated form of
  citation, omitting the article title, as follows.

  \begin{itemize}
  \tightlist
  \item
    (P.\textsubscript{S.}Satheshkumar, A.\textsubscript{S.}Weisberg, and
    B.\textasciitilde Moss, J Virol 87:10700--10709, 2013,
    doi:10.1128/JVI.01258-13)
  \item
    (J.\textsubscript{H.}Coggin, Jr., p.\textasciitilde93--114, in
    D.\textsubscript{O.}Fleming and D.\textsubscript{L.}Hunt, ed.,
    \emph{Biological Safety. Principles and Practices,} 4th ed.,

    \begin{enumerate}
    \def\labelenumii{\arabic{enumii})}
    \setcounter{enumii}{2005}
    \tightlist
    \item
    \end{enumerate}
  \end{itemize}

  \begin{quote}
   ``\ldots  in a recent report by D. A. Hopwood (mBio 4:e00612-13, 2013, doi:10.1128/mBio.00612-13) \ldots''
   \end{quote}
\item
  \textbf{References related to supplemental material.} If references
  must be cited in the supplemental material, list them in a separate
  References section within the supplemental material and cite them by
  those numbers; do not simply include citations of numbers from the
  reference list of the associated article. If the same reference(s) is
  to be cited in both the article itself and the supplemental material,
  then that reference would be listed in both References sections.
\item
  \textbf{Citations of data sets and/or code.} To encourage data sharing
  and reuse, ASM recommends reporting data sets and/or code both in a
  dedicated ``Data availability'' paragraph and in the reference list.
  Examples related to data citation have been reproduced in the template
  bib file, as have other sample references.
\end{enumerate}

\textbf{Author Warranty.} If accepted for publication, the Work will be
made freely available to the public on ASM's
\href{https://journals.asm.org/journal/msystems}{\emph{mSystems}}
website. ASM will grant the public the nonexclusive right to copy,
distribute, adapt, and transmit the published Work for commercial or
non-commercial use with proper attribution under the Creative Commons,
Attribution license, Version 4.0 (CC-BY). For details, see
\url{https://creativecommons.org/licenses/by/4.0/} and
\url{https://creativecommons.org/licenses/by/4.0/legalcode}, as well as
\url{https://journals.asm.org/author-warranty-and-provisional-license-publish}.

\LaTeX{} files. Authors who prefer to use \LaTeX{} for manuscript
preparation may use this Overleaf template. Files from an Overleaf
project may be transferred directly from Overleaf to mSystems once for
initial manuscript submission. Although a compiled PDF alone is
acceptable for initial submission of a manuscript prepared in \LaTeX{},
mSystems requires all \LaTeX{} files from a project to be uploaded at
the revision stage. On the mSystems submission site, the .tex file
should be classified as a Manuscript Text File. Other supporting files
that appear in the Overleaf package (i.e., .bib, .bst, .cls, .ldf, and
.sty files) should all be included and classified as `LaTeX Support
Files'. Figure files should be classified as Figure files; the usual
formatting restrictions apply (see mSystems Instructions to Authors).
When the mSystems manuscript record already exists, files must be
replaced on the mSystems submission site
(\url{https://msystems.msubmit.net}) if modification is required.
Contact journal staff with questions related to file conversion in the
mSystems manuscript record.

\section{Results}\label{results}

In the Results section, include the rationale or design of the
experiments as well as the results; reserve extensive interpretation of
the results for the Discussion section. Present the results as concisely
as possible in one or more of the following: text, table(s), or
figure(s). Data in tables (e.g., cpm of radioactivity) should not
contain more significant figures than the precision of the measurement
allows. Illustrations (particularly photomicrographs and electron
micrographs) should be limited to those that are absolutely necessary to
show the experimental findings. Number figures and tables in the order
in which they are cited in the text, and be sure to cite all figures and
tables. Figure~\ref{fig-example} is just for show, but this sentence
shows how a figure could be cited in the text of the manuscript.

The tabularx, booktabs and siunitx packages are loaded by
asm-article.cls; see \autoref{tab:example} for an example table. Use
\verb|\begin{fullwidth}...\end{fullwidth}| in your table for the table
to span the entire width of the page. Shading in the field of tables is
allowed, to demonstrate relationships among data. You can use the
\verb|\columncolor|, \verb|\rowcolor| or \verb|\cellcolor| commands to
do this: allowed color values are \verb|black!20| and \verb|black!30|.

\subsection{File types and formats.}\label{file-types-and-formats.}

Illustrations may be continuous-tone images, line drawings, or
composites. On initial submission, illustrations may be supplied as PDF
files, with the legend on the same page, to assist review. At the
modification stage, production quality digital files must be provided,
along with text files for the legends. The legends are copyedited and
typeset for final publication, not included as part of the figure
itself.

All graphics submitted with modified manuscripts should be grayscale or
in the RGB color mode. Minimum resolution is 300 dpi for all file types.
All images imported into a figure file must be at the correct resolution
before they are placed in the file. (For instance, placing a 72-dpi
image in a 300-dpi EPS file will not result in the placed image meeting
the minimum requirements for file resolution.) Note that publication
quality will not be improved by using a resolution higher than the
minimum.

All graphics should be submitted at their intended publication size;
that is, the image uploaded should be 100\% of its print dimensions so
that no reduction or enlargement is necessary. Resolution must be at the
required level at the submitted size. Include only the significant
portion of an illustration. White space must be cropped from the image,
and excess space between panel labels and the image must be eliminated.

\begin{itemize}
\tightlist
\item
  Maximum figure width: 6.875 inches (ca.\textasciitilde17.4 cm)
\item
  Maximum figure height: 9.0625 inches (23.0 cm)
\end{itemize}

\section{Discussion}\label{discussion}

The Discussion section should provide an interpretation of the results
in relation to previously published work and to the experimental system
at hand and should not contain extensive repetition of the Results
section or reiteration of the introduction. In short papers, the Results
and Discussion sections may be combined.

\begin{equation}
\frac{\partial^2 \Phi}{\partial x^2} + \frac{\partial^2 \Phi}{\partial y^2} +
            \frac{\partial^2 \Phi}{\partial z^2} =
            \frac{1}{c^2}\frac{\partial^2\Phi}{\partial t^2}
\end{equation}

Please note that display equations in the Overleaf template may be
rendered with a slightly different presentation in the final published
(\emph{mSystems}) article.

Lorem ipsum dolor sit amet, consectetur adipiscing elit, sed do eiusmod
tempor incididunt ut labore et dolore magna aliqua. Ut enim ad minim
veniam, quis nostrud exercitation ullamco laboris nisi ut aliquip ex ea
commodo consequat. Duis aute irure dolor in reprehenderit in voluptate
velit esse cillum dolore eu fugiat nulla pariatur. Excepteur sint
occaecat cupidatat non proident, sunt in culpa qui officia deserunt
mollit anim id est laborum.

\begin{equation}
\int_0^\infty e^{-\alpha x^2} \mathrm{d}x =
            \frac12\sqrt{\int_{-\infty}^\infty e^{-\alpha x^2}}
            \mathrm{d}x\int_{-\infty}^\infty e^{-\alpha y^2}\mathrm{d}y =
            \frac12\sqrt{\frac{\pi}{\alpha}}
\end{equation}

Lorem ipsum dolor sit amet, consectetur adipiscing elit, sed do eiusmod
tempor incididunt ut labore et dolore magna aliqua. Ut enim ad minim
veniam, quis nostrud exercitation ullamco laboris nisi ut aliquip ex ea
commodo consequat.

\section{Materials and Methods}\label{materials-and-methods}

The Materials and Methods section should include sufficient technical
information to allow the experiments to be repeated. When centrifugation
conditions are critical, give enough information to enable another
investigator to repeat the procedure: make of centrifuge, model of
rotor, temperature, time at maximum speed, and centrifugal force
(\(\times g\) rather than revolutions per minute). For commonly used
materials and methods (e.g., media and protein concentration
determinations), a simple reference is sufficient. If several
alternative methods are commonly used, it is helpful to identify the
method briefly as well as to cite the reference. For example, it is
preferable to state
\texttt{cells\ were\ broken\ by\ ultrasonic\ treatment\ as\ previously\ described\ (9)\textquotesingle{}\textquotesingle{}\ rather\ than\ to\ state}cells
were broken as previously described (9).'\,' This allows the reader to
assess the method without constant reference to previous publications.
Describe new methods completely and give sources of unusual chemicals,
equipment, or microbial strains. When large numbers of microbial strains
or mutants are used in a study, include tables identifying the immediate
sources (i.e., sources from whom the strains were obtained) and
properties of the strains, mutants, bacteriophages, and plasmids, etc.

A method or strain, etc., used in only one of several experiments
reported in the paper may be described in the Results section or very
briefly (one or two sentences) in a table footnote or figure legend. It
is expected that the sources from whom the strains were obtained will be
identified.

\subsection{Availability of data and
materials.}\label{availability-of-data-and-materials.}

By publishing in mSystems, the authors agree that, subject to
requirements or limitations imposed by local and/or U.S. Government laws
and regulations, any materials and data that are reasonably requested by
others are available from a publicly accessible collection or will be
made available in a timely fashion, at reasonable cost, and in limited
quantities to members of the scientific community for noncommercial
purposes. Similarly, the authors agree to make available computer
programs and/or code, originating in the authors' laboratory, that is
the only means of confirming the conclusions reported in the article but
that is not available commercially. The program(s) and suitable
documentation regarding its (their) use may be provided by any of the
following means: (i) as a program transmitted via the Internet, (ii) as
an Internet server-based tool, or (iii) as a compiled or assembled form
on a suitable medium. The authors guarantee that they have the authority
to comply with this policy either directly or by means of material
transfer agreements through the owner. ASM asks authors to assert this
in a ``Data availability'' paragraph, which should appear at the end of
the Materials and Methods section (or at the end of the text) of their
submitted manuscript.

Therefore, a condition of publication in mSystems is that authors make
data fully available and without restriction, except in rare
circumstances. Data availability will be confirmed prior to publication
and must be provided during the modification stage, if not before.
Furthermore, data must be made available, upon request, for peer review.
See \href{https://journals.asm.org/open-data-policy}{Data Policy}.

\subsection{Data citation.}\label{data-citation.}

To promote reproducibility, ASM expects researchers to identify and cite
data sets and/or code used in their experiments and studies. These may
be large or complex data sets that can include, but are not limited to,
data from microarray, genomic, structural, proteomic, or video imaging
analyses. \textbf{Authors should cite both the data set repository and
the published article in which the data set and/or code was originally
described.} Citations of data should be included in the reference list
with persistent unique identifiers (e.g., active DOIs, accession
numbers, etc.). If computer code or software was created to generate
results or interpret data, then a statement to that effect should be
included in the ``Data availability'\,' paragraph. For cases in which
the software is publicly available (e.g.,
\href{http://tree.bio.ed.ac.uk/software/figtree/}{FigTree} to generate
phylogenetic trees), the URL of the software informational page should
be provided. \textbf{It is preferred that authors use established,
publicly available}
\href{https://journals.asm.org/list-data-repositories}{data type-specific repositories}.
If there is no appropriate repository available, general publicly
available repositories should be used (e.g.,
\href{https://datadryad.org/stash}{Dryad},
\href{https://figshare.com/}{figshare}, etc.).

\section{Supplemental Material}\label{supplemental-material}

Guidelines for supplemental material appear in the Instructions to
Authors. This section of the paper should include legends for any
supplemental material that is intended for posting. Such supplemental
material must be submitted with the manuscript. Files can be added to
the submission at the publisher's submission site. Here is a list of
sample legends for supplemental material:

\textbf{FIG S1.} Supplemental file 1 is a figure that shows results
related to the study, although the study stands on its own. This legend
for the figure may include multiple sentences.

\textbf{FIG S2.} Supplemental file 2 is a figure that shows results
related to the study, although the study stands on its own. This legend
for the figure may include multiple sentences.

\textbf{FIG S3.} Supplemental file 3 is a figure that shows results
related to the study, although the study stands on its own. This legend
for the figure may include multiple sentences.

\textbf{TABLE S1.} Supplemental file 4 is a large table that shows
results related to the study, although the study stands on its own. This
legend for the table may include multiple sentences.

\textbf{TABLE S2.} Supplemental file 5 is a complex table that shows
results related to the study, although the study stands on its own. This
legend for the supplemental material may include multiple sentences.

\section{Acknowledgments}\label{acknowledgments}

Statements regarding sources of direct financial support (e.g., grants,
fellowships, and scholarships, etc.) should appear in the
Acknowledgments. A funding statement indicating what role, if any, the
funding agency had in your study (for example,
\texttt{The\ funders\ had\ no\ role\ in\ study\ design,\ data\ collection\ and\ interpretation,\ or\ the\ decision\ to\ submit\ the\ work\ for\ publication.\textquotesingle{}\textquotesingle{})\ may\ be\ included.\ Funding\ agencies\ may\ have\ specific\ wording\ requirements,\ and\ compliance\ with\ such\ requirements\ is\ the\ responsibility\ of\ the\ author.\ In\ cases\ in\ which\ research\ is\ not\ funded\ by\ any\ specific\ project\ grant,\ funders\ need\ not\ be\ listed,\ and\ the\ following\ statement\ may\ be\ used:}This
research received no specific grant from any funding agency in the
public, commercial, or not-for-profit sectors.'\,' Statements regarding
indirect financial support (e.g., commercial affiliations,
consultancies, stock or equity interests, and patent-licensing
arrangements) are also allowed. It is the responsibility of authors to
provide a general statement disclosing financial or other relationships
that are relevant to the study. Recognition of personal assistance
should be given as a separate paragraph, as should any statements
disclaiming endorsement or approval of the views reflected in the paper
or of a product mentioned therein. In addition to acknowledging sources
of financial support in the manuscript, authors should list any sources
of funding in response to the Funding Sources question on the online
submission form, providing relevant grant numbers where possible, and
the authors associated with the specific funding sources. In the event
that your submission is accepted, the funding source information
provided in the submission form may be published, so please ensure that
all information is entered accurately and completely. (It will be
assumed that the absence of any information in the Funding Sources
fields is a statement by the authors that no support was received.)

Authors may include a statement that specifies contributor roles as a
separate paragraph in the Acknowledgments section. ASM encourages
transparency in authorship by publishing author contribution statements
using the CRediT taxonomy as recommended by
\href{https://casrai.org/credit/}{CASRAI}. For some manuscript types,
authors have the option of assigning CRediT roles during the online
submission process.

\textbf{Please read the
\href{https://journals.asm.org/journal/msystems/submission-review-process}{Instructions
to Authors} carefully, or browse the
\href{https://journals.asm.org/journal/msystems/faq}{FAQs} for further
details.}


  \bibliography{bibliography.bib}



\end{document}
